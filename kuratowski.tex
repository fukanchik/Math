\documentclass[12pt,french]{article}

\usepackage[frenchb]{babel}
\usepackage[utf8]{inputenc}
\usepackage[T1]{fontenc}
\usepackage{amssymb}
% \usepackage[perpage]{footmisc}

\title{Sur la notion de l'ordre dans la Th\'eorie des Ensembles.}

\date{}

\author{Par
Casimir Kuratowski (Varsovie)}

\begin{document}
\maketitle

Les consid\'erations qui sont expos\'ees dans cette Note suivent la voie des id\'ees des M.M. Hessenberg\footnotemark{} et Hartogs\footnotemark{} sur la m\'ethode d'introduction de la notion d'ordre dans la Th\'eorie des Ensembles.

Cette m\'ethode peut \^etre resum\'ee, comme suit.

Soit $M$ un ensemble quelconque ; convenons de dire que la classe\footnotemark{} $M$ \textit{,,\'etablit un ordre''} dans l'ensemble $M$ lorsqu'elle v\'erifie les conditions suivantes:

\begin{description}
  \item[(1)] les \'el\'ements de la classes $M$ sont des sous-ensembles de $M$;
  \item[(2)] $X$ et $Y$ \'etant des \'el\'ements quelconques de $M$ on a: \[X \subset Y \textrm{ ou bien } Y \subset X;\]
\end{description}

\footnotetext{\textit{Grundbegriffe der Mengenlehre.} Abhandlungen der Fries'schen Schule I, 4,G\"ottingen 1906, p. 674--685 (,,Vollst\"andige ordnende Systeme'').

Dans une note publi\'e \`a la m\^eme \'epoque (\textit{Sur les \'el\'ements de la th\'eorie des ensembles ordonn\'es}, Enseignement Math\'ematique VIII, 1906 Mai--Juin, p. 201) \texttt{M. Comb\'ebiac} a exprim\'e sur la th\'eorie de l'ordre des id\'ees bien analogues \`a celles de \texttt{M. Hessenberg}.}

\footnotetext{\textit{Ueber das Problem der Wohlordnung}. Anbang. Mathematische Annalen 76, 1914, p. 443.}

\footnotetext{Pour la commodit\'e du la langage nous ferons usage du terme ,,classe'' lorsqu'il sera question des ensembles, dont les \'el\'ements sont eux-m\^emes des ensembles par hypoth\'ese; nous d\'esignons les classes par $A$, $B$, $C$... Leurs \'el\'ements-ensembles par $A$, $B$ ...; les \'el\'ements de ces derniers par $a$, $b$ ...

$$a \in A$$

signifie, comme d'habitude que $a$ est un \'el\'ement de $A$;

$$A \subset B$$

signifie que $A$ est un sous-ensemble de $B$, que $A$ est contenu dans $B$.}

\begin{description}
  \item[(3)] $x$ et $y$ \'etant deux \'el\'ements diff\'erents de $M$, il existe un ensemble-\'el\'ement de $M$ qui en contient un sans en contenir l'autre;
  \item[(4)] $X$ \'etant une sous-classe de $M$, la somme du tous les ensembles qui sont les \'el\'ements de $X$, est un \'el\'ement de $M$;
  \item[(5)] il est de m\^eme du produit de ces ensembles.
\end{description}

On d\'emontre, que lorsque $M$ remplit les conditions 1--5 et lorsqu'on pose

\[ x < y \]

quand il existe un \'element $Y$ de $M$ dont $y$ est un \'el\'ement et $x$ ne l'est pas, -- l\'ensemble $M$ est \textit{ordonn\'e} au sens habituel du mot. D'autre part, lorsqu'on suppose que l'ensemble $M$ est ordonn\'e d'une certaine fa\c con, la classe de tous ses restes\footnotemark{}  v\'erifie les conditions 1--5; elle est d'ailleurs la seule\footnotemark{} qui les v\'erifie en l'ordonnant de cette fa\c con. Donc, il existe une correspondance biunivoque entre les fa\c cons d'ordonner un ensemble donn\'e et les classes qui y ,,\'etablissent un ordre''.

Ainsi, la th\'eorie des \texttt{classes qui \'etablissent un ordre} peut \^etre regard\'ee comme \'equivalente \`a la the\'eorie classique des ensembles ordonn\'es, bas\'ee sur la notion intuitive d'ordre (\texttt{Cantor}). En m\^eme temps, on peut la d\'eduire de la th\'eorie g\'en\'erale des ensembles (non ordonn\'es) sans qu'il y faille introduire aucune notion premi\'ere suppl\'ementaire: l'id\'ee de l'orre y est donn\'ee en termes fondamentauxdus syst\'eme des axiomes de \texttt{M.Zermelo}\footnotemark{}, \`a savoir, celui d'ensemble et celui d'\'el\'ement\footnotemark{}. L'importance de cette m\'ethode est manifeste.


\footnotetext{On apelle ,,reste'' d'un ensemble ordonn\'e tout ensemble qui jouit de la propri\'et\'e suivante: lorsque $x$ est son \'el\'ement, tout \'el\'ement pr\'ec\'ed\'e par $x$ l'est aussi. (\texttt{Hessenberg, l. c. p. 541}).}

\footnotetext{\texttt{M. Hartogs} affirme, ue $M$ \'etant un ensemble ordonn\'e d'une certaine fa\c con et $M$ une classe l'ordonnant ainsi et ne satisfaisant qu'aux conditios 1--4, -- $M$ est identique \`a la classe de tous los reses de $M$ (p. 443, lignes: 12--15).

Ceci est inexact. En effet, soit $M$ l'ensemble de tons les $x$ satisfaisant \`a l'in\'egalit\'e: $0  \leqslant x \leqslant 1$ ; envisageons la classe $S$ de tous les segments contenus dans $M$ contenant le point $1$; soit $M_1$ la classe de tous les ensembles de }

\footnotetext{a}

\footnotetext{b}

\end{document}
